\documentclass{article}
\usepackage[utf8]{inputenc}

\title{Variables}
\author{Arsh Suri}
\date{July 2021}

\begin{document}

\maketitle

\section{Defining the Signal Time Series}
The Signal Time Series is the predicted time-dependent particle event rate of 
Heavy Neutral Leptons (HNLs), $N$, decay. This predicted signal should be at its minimum at Mid-day and its max at midnight due to the rotation of the Earth. This modulation, for our purposes, 
can be represented as a sinusoidal wave whose amplitude is dictated by the HNLs mass, $M_{N}$, 
and its dipole coupling strength, $d_{N}$.

\section{Defining the Background Time Series}
The Background Time Series is the total resulting time-independent event rate of the different nuclear reactions, beta decays, and others producing neutrinos. This background should be relatively constant throughout time.  

\section{Computing Expected Background Rate}
To compute the Expected Background rate, it is the sum of the different background events per
day per a 100T per a $N_{h}$. Representing the different backgrounds as $Bkg_{0}(N_{h})$, 
$Bkg_{1}(N_{h})$, $Bkg_{2}(N_{h})$, ... $Bkg_{n}(N_{h})$. For instance the predicted events per day for polonium 210 at 200 Hits can be $Bkg_{0}(200)$. By setting a max and min number of hits, $H_{min}$ and 
$H_{max}$ this background rate can be represented as

\begin{equation}
   \lambda_{bkg} = \sum_{i=0}^{n} \int_{H_{min}}^{H_{max}} Bkg_{i}(N_{h}) dN_{h}
\end{equation}

where $n$ is the number of backgrounds. Then we can generate samples using a Poisson Distribution with the mean of $\lambda_{bkg}$.

\section{Computing Expected Signal Rate in the Off and On Bin}
We can set up a sine wave with the formula

\begin{equation}
    sig(t) = A\sin\left(\frac{\pi}{12}\left(t+6\right)\right)+A 
\end{equation}
where A is the amplitude determined by $d_{N}$ and $M_{N}$. In this formula, the time is measured in hours and it will give us the expected event rate. We can represent the expected signal event rate in the 12 hour off-bin as

\begin{equation}
  \lambda_{off} = \int_{0}^{12} [A\sin\left(\frac{\pi}{12}\left(t+6\right)\right)+A] dt
\end{equation}

and the expected number of events in the 12 hour on-bin as

\begin{equation}
   \lambda_{on} = \int_{12}^{24} [A\sin\left(\frac{\pi}{12}\left(t+6\right)\right)+A] dt
\end{equation}

\section{Generating Observable}
These computed values can allow us to generate observable. For the off-bin, the observable number of events can be calculated as

\begin{equation}
    P_{off} = Pois(\lambda_{bkg}) + Pois(\lambda_{off})
\end{equation}

Similarly, the observable number of events in the on-bin can be calculated as 
\begin{equation}
    P_{on} = Pois(\lambda_{bkg}) + Pois(\lambda_{on})
\end{equation}

\end{document}
